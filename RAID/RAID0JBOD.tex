\subsection{RAID 0 - JBOD}
Předpokládejte následující konfiguraci diskového pole:

\begin{itemize}
    \item Typ disků: Seagate Barracuda 7200.14 (SAS, 6 Gib/s)
    \item Počet a velikost disků: 14 x 3000 GiB
    \item Rychlost disku: 10000 rpm, 80 MiB/s
    \item Spolehlivost disků: 865000 h MTBF
    \item Typ diskového pole: RAID JBOD, 1 řadič pracující s rychlostí 400 MiB/s
    \item Připojení pole: Ethernet 1000BASE-T (1 Gbit/s) x 1 rozhraní
\end{itemize}


\textbf{Jakou bude mít takové pole kapacitu (v GiB)?}

RAID JBOD je jednoduché sériové zapojení. Kapacita $14*3000 = 42000GiB$

\textbf{Jakou rychlostí bude v ideálním případě možné z pole sekvenčně číst (v MiB/s)? Odpověď má povolenou toleranci 1\%.}

Rychlost čtení jednoho disku je 80, rychlost řadiče 400. Vždy čteme z jednoho disku. 80.

\textbf{Jakou rychlostí bude v ideálním případě možné na pole sekvenčně zapisovat (v MiB/s)? Odpověď má povolenou toleranci 1\%.}

Rychlost zápisu jednoho disku je 80, rychlost řadiče 400. Vždy zapisujeme na jeden disk. 80.

\textbf{Kolik disků může určitě vypadnout, aniž dojde ke ztrátě dat? Uvažujte nejhorší možný scénář výpadků.}

Všechny diksy jsou nenahraditelné. 0.

\textbf{Jaká je pravděpodobnost (v procentech), že během 36 měsíců nedojde ke ztrátě dat \\(předpokládáme, že během této doby v poli nic nevyměňujeme)?}

Doba, za kterou chceme měřit, je $36$ měsíců, neboli ${36 \over 12}$ let : ${36 \over 12}$ let $* 365$ dní $* 24$ hodin. MTBF je $865 000$. 

Pro jeden disk vychází: 
$$D = e^{-{{36 \over 12}*365*24 \over 865000}} =  0.9700753764\%$$

V RAIDu máme 14 disků, umocníme:
$$D^{14} = 0.6535468612 = 65.35\%$$

\textbf{Kolik sekund bude trvat obnova dat při obnově jednoho disku za předpokladu, že uživatel diskové pole trvale zatěžuje požadavky průměrně 20 MiB/s? Pokud obnova nelze provést, ponechte pole prázdné. Odpověď má povolenou toleranci 1\%.}

Obnovit RAID JBOD nejde.