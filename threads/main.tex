\section{Plánování vláken}

\subsection{Příklad 1}
\subsubsection{Zadání}

Předpokládejme následující systém:
\begin{itemize}
    \item Procesor(y): 1 x Intel® Core™ i5-7360U (počet jader na procesor: 2)
    \item Počet vláken na jádro: 2 (tedy celkem vláken na procesor: 4)
    \item Architektura: SMP/UMA (desktop)
    \item Paměť: 4 GiB DDR3
\end{itemize}

Na systému je nainstalovaný 64-bitový operační systém, který používá pro plánování uživatelských procesů statickou prioritu (vyšší hodnota znamená vyšší prioritu). Jádro/procesor se přiděluje jednotlivým vláknům maximálně na dobu 20 ms a režii přepínání kontextu zanedbáváme. Dále předpokládáme:

\begin{itemize}
    \item jádro a ostatní procesy (kromě $P_0, ..., P_3$ ) zatěžují procesor z $20\%$
    \item plánovač se snaží prodloužit životnost CPU tím, že udržuje rovnoměrnou teplotu všech jader, tedy výpočetní zátěž rovnoměrně rozkládá na jednotlivá jádra
    \item pro každého běžného uživatele je nastaven limit, který dovolí uživateli spustit maximálně 640 vláken, po dosažení limitu budou funkce vytvářející nová vlákna vracet chybu
    \item všichni běžní uživatele z následující tabulky spustí své procesy v čase $T_0$
    \item u každého procesu je uvedena statická priorita, počet vláken, která provádějí výpočet, a celková doba výpočtu (doba výpočtu procesu, pokud by běžel sekvenčně na jednom nezatíženém jádru). Pokud je výpočet v procesu realizován pomocí více vláken na dostatečném počtu jader, pak předpokládáme lineární zrychlení výpočtu, a režii na vytváření a synchronizaci vláken zanedbáme
\end{itemize}

\begin{table}[h]
    \centering
    \begin{tabular}{ l|l|l|l|l }
        \hline
        Uživatel & Proces & Priorita & Počet vláken & Celková doba výpoču (sec)\\
        \hline
        zahradn & $P_0$ & 90 & 1 & 240\\
        zahradn & $P_1$ & 60 & 3 & 900\\
        zahradn & $P_2$ & 60 & 1 & 180\\
        zdarekj & $P_3$ & 10 & 2 & 600\\
        \hline
    \end{tabular}
\end{table}

\textbf{Za kolik sekund po $T_0$ se ukončí proces $P_2$? (Výsledek zadejte i s desetinnou částí, je tolerovaná malá odchylka.)}

\textbf{Za kolik sekund po $T_0$ se fakticky začne zpracovávat proces $P_3$ ? (Výsledek zadejte i s desetinnou částí, je tolerovaná malá odchylka.)}

\textbf{Za kolik sekund po $T_0$ se ukončí poslední proces? (Výsledek zadejte i s desetinnou částí, je tolerovaná malá odchylka.)}

\subsubsection{Analýza}
\textbf{Informace ze zadání:}
\begin{itemize}
    \item Počet dostupných vláken je $4$.
    \item Systém je již zatížen na $20\%$. Na konci budeme muset upravit všechny spočítané časy v závislosti na zátěži.
    \item Všimneme si tabulky procesů, jejich priority, počtu vláken a celkové doby výpočtu. Pozor, doba běhu jednoho procesu je uvedena tak, jako kdyby běžela pouze na jednom vláknu. Použitím více vláken se doba zrychlí.
\end{itemize}

V průběhu výpočtu nastanou situace, kdy procesy budou mít stejné priority a jejich potřeby z hlediska vláken budou převyšovat aktuálně dostupný počet vláken. Například budou dvě vlákna, první bude potřebovat 3 vlákna, druhé pouze 1. Dostupná budou pouze 3 vlákna. Po výpočet to bude znamenat, že 3 vlákna se budou dělit v poměru $3:1$.

Obecný vzorec pro výpočet bude vypadat následovně:
$$T_N = T_P/ {{process\ threads} \over {thread\ sum}}/available$$

\begin{itemize}
    \item $T_P$ je čas, který proces potřebuje, $T_N$ je čas, který ve skutečnosti poběží s daným poměrem a počtem dostupných vláken 
    \item $process\ threads$ je počet vláken, který daný proces chce
    \item $thread\ sum$ je součet počtu vláken, které aktuálně chtějí počítat ($3 + 1 = 4$)
    \item $available$ je počet aktuálně dostupných vláken
\end{itemize}

\subsubsection{Výpočet}
Výpočet je založen na postupným přiřazování vláken procesům podle priority. Pokud nějaký proces doběhne, tak "pozastavíme" výpočet, spočítáme již provedenou práci všech paralelně pracujících vláken a pokračujeme s novým přerozdělením vláken.

V čase $T_0$ se spustí $P_0$ (má nejvyšší prioritu) a dostane 1 vlákno (víc nechce). Zbylá tři vlákna se rozdělí mezi $P_1$ a $P_2$.

$P_0$ poběží 240s bez přerušení dokud neskončí. $P_1$ a $P_2$ si musí rozdělit výkon 3 vláken v poměru $3:1$ (počet vláken $P_1$ : počet vláken $P_2$).

Spočítáme teoretickou dobu běhu pro $P_1$ a $P_2$ - s aktuálním rozdělením vláken procesy poběží tuto dobu:
$$P_1: 900/{3 \over 4}/3=400$$
$$P_2: 180/{1 \over 4}/3=240$$

$P_2$ poběží stejně dlouho jako $P_0$ a oba procesy v čase $T_0+240$ dokončí práci a uvolní $1+1=2$ vlákna. V tu chvíli pozastaví $P_1$ své výpočty, abychom přerozdělili vlákna.

Musíme spočítat kolik práce již vykonala $P_1$ během prvních $240s$. Vynásobíme dobu běhu bývalým poměrem a počtem volných vláken.
$$P_1: 240*{3 \over 4}*3 = 540s$$

$P_1$ Během prvních $240s$ se vykonalo prvních $540s$ práce. Zbývá $900-540=360s$ ($340s$ pro \textit{jedno} vlákno).

Jelikož má $P_1$ aktuálně nejvyšší prioritu, tak si zabere všechna 3 potřebná vlákna a bude pokračovat ve výpočtu. Zbylé vlákno dostane čekající $P_3$.

$$P_1: 360/{3 \over 3}/3=120$$
$$P_3: 600/{2 \over 2}/1=600$$

$P_1$ skončí po $T_0+240+120$, do té doby bude muset $P_3$ pracovat pouze na jednom vláknu. Po doběhnutí $P_1$ musíme opět spočítat kolik práce vykonala $P_3$ a kolik zbývá.

$$P_3: 600 - 120*{2 \over 2}*1 = 600 - 120 = 480s$$

Všechna vlákna jsou volná, přiřadíme potřebný počet $P_3$ (2).
$$P_3: 480/{2 \over 2}/2 = 240$$

$P_3$ doběhne po $T_0 + 240 + 120 + 240$. Všechny procesy jsou dokončeny.

\textbf{Stručný běh vláken a mezivýsledky:}

\begin{itemize}
    \item $P_0$ se spustilo rovnou, běželo neustále na jednom vlákně
    \item $P_1$ a $P_2$ se spustili rovnou a dělili mezi sebou 3 vlákna
    \item $P_0$ a $P_2$ se dokončily současně na 240sekundě. Dvě vlákna se uvolnily.
    \item $P_1$ už může běžet na 3 vláknech. Spouští se $P_3$ na jednom vlákně (240s po startu).
    \item $P_1$ doběhne po dalších 120s (celkově na 360s). $P_3$ ted může běžet na 2 vláknech.
    \item $P_3$ doběhne po dalších 240s (celkově na 600s).
\end{itemize}

\begin{itemize}
    \item $P_1$ a $P_2$ doběhnou po $T_0 + 240$
    \item $P_1$ doběhne po $T_0 + 240 + 120$
    \item $P_3$ doběhne po $T_0 + 240 + 120 + 240$
\end{itemize}

Systém ale není úplně volný, je zatížen jádrem na $20\%$.

Pro všechny časy spočítáme novou hodnotu přes jednoduchou trojčlenku. Příklad výpočtu pro 240s:
$${x \over 240} = {100 \over {100 - 20}} \Rightarrow x = {24000 \over 80} = 300$$


\begin{table}[h]
    \centering
    \begin{tabular}{ l|l|l }
        \hline
        Proces & Začátek (s vytížením) & Konec (s vytížením)\\
        \hline
        $P_0$ & 0 & 240 (300)\\
        $P_1$ & 0 & 360 (450)\\
        $P_2$ & 0 & 240 (300)\\
        $P_3$ & 240 (300) & 600 (750)\\
        \hline
    \end{tabular}
\end{table}

\textbf{Za kolik sekund po $T_0$ se ukončí proces $P_2$? (Výsledek zadejte i s desetinnou částí, je tolerovaná malá odchylka.)} 300

\textbf{Za kolik sekund po $T_0$ se fakticky začne zpracovávat proces $P_3$ ? (Výsledek zadejte i s desetinnou částí, je tolerovaná malá odchylka.)} 300

\textbf{Za kolik sekund po $T_0$ se ukončí poslední proces? (Výsledek zadejte i s desetinnou částí, je tolerovaná malá odchylka.)} 750