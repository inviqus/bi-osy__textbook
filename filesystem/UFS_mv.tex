\subsection{UFS - Přesun souboru}

\textit{\textbf{Pozor, výjimka: pokud se přesun souboru provádí v rámci dvou odlišných file systemu, tak se chová jako kopírování!}}

\begin{minted}{shell}
mv /usr/local/opt/file.pdf /usr/local/opt/folder/new.pdf
\end{minted}

Musíme načíst obě cesty k souborům.
\begin{verbatim}
read i-node  /
read content /
read i-node  /usr
read content /usr
read i-node  /usr/local
read content /usr/local
read i-node  /usr/local/opt
read content /usr/local/opt # nutné pro zjištění i-node souboru
\end{verbatim}

Cílová cesta je pouhé pokračování původní, načteme pouze to, co zatím není v paměti. Content načítat nemusíme, protože ho nepotřebujeme.
\begin{verbatim}
read i-node  /usr/local/opt/folder
\end{verbatim}

Upravíme content cílové složky, protože má nový soubor, a i-node složky kvůli mtime.
\begin{verbatim}
write content /usr/local/opt/folder
write i-node /usr/local/opt/folder
\end{verbatim}

Upravíme content původní složky, protože přišla o soubor, a i-node složky kvůli mtime.
\begin{verbatim}
write content /usr/local/opt
write i-node /usr/local/opt
\end{verbatim}

\begin{itemize}
    \item \textbf{Kolik i-nodů se musí načíst?} 5 
    (5 složek)
    \item \textbf{Kolik i-nodů se musí zapsat?} 2
    (2 upravené složky)
    \item \textbf{Kolik datových bloků se musí načíst?} 5 
    (5 složek)
    \item \textbf{Kolik datových bloků se musí zapsat?} 1
    \item \textbf{Kolik operací musí být provedeno při alokování/uvolňování bloků a i-nodů? Každou operaci alokace/uvolnění bloku/i-node započtěte zvlášť.} 5183533
    (5183532 bloků odkazů a dat souboru, 1 i-node souboru)
\end{itemize}
