\subsection{UFS - Vytváření symlinku}

\begin{minted}{shell}
ln -s /usr/local/opt/file.pdf /tmp/user/file.link
\end{minted}

Pro vytvoření symlinku nepotřebujeme pracovat s exitujícím souborem, stačí pouze jeho název a cesta\\ (\mintinline{shell}{/usr/local/opt/file.pdf}).
Nejdřív musíme načíst celou cestu k \mintinline{shell}{/tmp/user/file.link} souboru (včetně \mintinline{shell}{/}, protože není načtený do paměti).

\begin{verbatim}
read i-node  /
read content /
read i-node  /tmp
read content /tmp
read i-node  /tmp/user
read content /tmp/user
\end{verbatim}

Vytvoříme v file systemu nový soubor (vytvářím i-node). Jeho jediným datovým blokem, na který bude ukazovat jeden přímý odkaz, bude blok s názvem cílového souboru.
\begin{verbatim}
write i-node /tmp/user/file.link
write content /tmp/user/file.link
\end{verbatim}

Změníme content parent složky (\mintinline{shell}{/tmp/user}), že ted obsahuje nový soubor. Protože jsme modifikovali složku, tak musíme upravit i-node složky kvůli údaji o poslední modifikaci prvku (mtime).
\begin{verbatim}
write content /tmp/user
write i-node /tmp/user
\end{verbatim}

\begin{itemize}
    \item 
        \textbf{Kolik i-nodů se musí načíst?} 3 
        - (\mintinline{shell}{read i-node}) - otevíráme složky
    \item 
        \textbf{Kolik i-nodů se musí zapsat?} 2 
        - (\mintinline{shell}{write i-node}) jednou jsme vytvořili soubor a jednou upravili čas modifikace složky
    \item 
        \textbf{Kolik datových bloků se musí načíst?} 3 
        - (\mintinline{shell}{read content}) otevíráme složky
    \item
        \textbf{Kolik datových bloků se musí zapsat?} 2
        - (\mintinline{shell}{write content}) jednou jsme vytvářeli soubor a jednou přidali do složky
    \item
        \textbf{Kolik operací musí být provedeno při alokování/uvolňování bloků a i-nodů? Každou operaci alokace/uvolnění bloku/i-node započtěte zvlášť.} 2
        - museli jsme vytvořit i-node pro nový soubor a pak pro něj alokovat jeden datový blok
\end{itemize}
